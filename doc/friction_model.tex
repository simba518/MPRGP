\documentclass[annual]{acmsiggraph}
\TOGonlineid{45678}
\TOGvolume{0}
\TOGnumber{0}
\TOGarticleDOI{1111111.2222222}
\TOGprojectURL{}
\TOGvideoURL{}
\TOGdataURL{}
\TOGcodeURL{}

\usepackage{amssymb}
\usepackage{xcolor}
\usepackage{algorithm}
\usepackage{algorithmic}
\usepackage{amsthm}
\renewcommand{\algorithmicrequire}{\textbf{Input:}}
\renewcommand{\algorithmicensure}{\textbf{Output:}}

\newcommand{\E}[1]{\mathbf{#1}}
\newcommand{\FPP}[2]{\frac{\partial{#1}}{\partial{#2}}}
\newcommand{\FPPTWO}[2]{\frac{\partial{#1}^2}{\partial^2{#2}}}
\newcommand{\FDD}[2]{\frac{d{#1}}{d{#2}}}
\newcommand{\TWO}[2]{\left(\setlength{\arraycolsep}{1pt}\begin{array}{cc}{#1} & {#2}\end{array}\right)}
\newcommand{\TWOC}[2]{\left(\begin{array}{c}#1 \\ #2\end{array}\right)}
\newcommand{\TWOR}[2]{\left(\setlength{\arraycolsep}{1pt}\begin{array}{cc}{#1}^T & {#2}^T\end{array}\right)^T}
\newcommand{\TWORC}[4]{\left(\begin{array}{cc}#1 & #2 \\ #3 & #4\end{array}\right)}
\newcommand{\THREEC}[3]{\left(\begin{array}{c}#1 \\ #2 \\ #3\end{array}\right)}
\newcommand{\THREER}[3]{\left(\setlength{\arraycolsep}{1pt}\begin{array}{ccc}{#1}^T & {#2}^T & {#3}^T\end{array}\right)^T}
\newcommand{\FOURC}[4]{\left(\begin{array}{c}#1 \\ #2 \\ #3 \\ #4\end{array}\right)}
\newcommand{\FOURR}[4]{\left(\setlength{\arraycolsep}{1pt}\begin{array}{cccc}{#1}^T & {#2}^T & {#3}^T & {#4}^T\end{array}\right)^T}
\newcommand{\SIXC}[6]{\left(\begin{array}{c}#1 \\ #2 \\ #3 \\ #4 \\ #5 \\ #6\end{array}\right)}
\newcommand{\SIXR}[6]{\left(\setlength{\arraycolsep}{1pt}\begin{array}{cccc}{#1}^T & {#2}^T & {#3}^T & {#4}^T {#5}^T {#6}^T\end{array}\right)^T}
\newcommand{\MTT}[4]{\left(\begin{array}{cc}#1 & #2 \\ #3 & #4\end{array}\right)}
\newtheorem{myDef}{Definition}
\newtheorem{myTheo}{Theorem}

\title{A Simple Friction Model}
\author{Siwang Li}
\pdfauthor{Siwang Li}

\begin{document}
\maketitle

\section{A Simple Friction Model}
Let's consider one point collides with a surface with normal $\E{n}$. Suppose the velocity of this point is $\E{v}$, and the external force is $\E{f}$, the static and kinetic friction coefficients are $\mu_s$ and $\mu_k$ respectively, and the mignitude for the normal pressure force is $\lambda$. Then, the friction force of this point is
\[
\E{f}_r = 
\left\{ \begin{array}{rl}
  -\min(\|\E{f}_t\|, \lambda \mu_k)\frac{\E{v}_t}{\|\E{v}_t\|}, &\E{v}_t \neq 0\\
  &\\
  -\min(\|\E{f}_t\|, \lambda \mu_s)\frac{\E{f}_t}{\|\E{f}_t\|}, & \E{v}_t = 0
\end{array} \right.\nonumber
\]
where $\E{f}_t$ and $\E{v}_t$ is calculated by using
\begin{eqnarray*}
 \E{f}_t &=& \E{f} - (\E{n} \cdot \E{f})\E{n}\\
 \E{v}_t &=& \E{v} - (\E{n} \cdot \E{v})\E{n}
\end{eqnarray*}
respectively. Finally, we need to update the force $\E{f}$ using $\E{f}_r$, i.e
\[
\E{f} = \E{f} + \E{f}_r
\]
{\color{red}{And when $\|\E{f}_t\| < \lambda \mu$, we also need to set $\E{v} = \E{v}-\E{v}_t$.}}

\bibliographystyle{acmsiggraph}
\bibliography{template}
\end{document}
